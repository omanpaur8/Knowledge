\section{INTEGRAL}


\subsection{Tabla de Integrales}
\begin{tabular}{ll}
	\hline \hline
	Función & Integral\\
	\hline
	$\#$ & $\#x+c$\\
	$x$ & $x^2+c$\\
	$x^n$ & $\dfrac{x^{n+1}}{n+1}+c$\\
	$\dfrac{1}{x}$ & $\ln|x|+c$\\
	$e^x$ & $e^x+c$\\
	$\#^x$ & $\dfrac{\#^x}{\ln(\#)}+c$\\
	$sin(x)$ & $-cos(x)+c$\\
	$cos(x)$ & $sen(x)+c$\\
	$sec^2(x)$ & $tan(x)+c$\\
	$sec(x)tan(x)$ & $sec(x)+c$\\
	$csc^2(x)$ & $-cot(x)+c$\\
	$csc(x)cot(x)$ & $-csc(x)+c$\\
	$tan(x)$ & $\ln|sec(x))|+c$\\
	$sec(x)$ & $\ln|sec(x)+tan(x)|+c$\\
	$\dfrac{1}{1+x^2}$ & $arctan(x)+c$\\
	$\dfrac{1}{\sqrt{1-x^2}}$ & $arcsen(x)+c$\\
	$\dfrac{1}{x\sqrt{x^2-1}}$ & $arcsec(x)+c$\\
	$\dfrac{1}{ax+b}$ & $\dfrac{\ln|ax+b|}{a}+c$\\
	$sin(\#x)$ & $\dfrac{-cos(\#x)}{\#}+c$\\
	$\dfrac{1}{\#+x^2}$ & $\dfrac{1}{\sqrt{\#}}arctan\left( \dfrac{x}{\sqrt{\#}}\right) +c$\\
	\hline
\end{tabular}\\

\subsection{Sustitución}
Es un cambio de variable para hacer la integral más fácil.
\begin{enumerate}
	\item Para escoger u, $u=algo$
	\begin{itemize}
		\item $arc\underline{\ \ \ }$
		\item $\sqrt{algo}$
		\item $e^{algo}$
		\item $(algo)$
		\item $\dfrac{1}{algo}$
		\item u nunca puede ser solo x o un $\#$
	\end{itemize}
	\item Derivar u
	\item Despejar dx
	\item Reemplazar en la integral y cancelar. Si no se cancela:
	\begin{itemize}
		\item La sustitución está mal o
		\item Toca despejar de la sustitución original.
	\end{itemize}
	\item Resolver
	\item Reemplazar u
\end{enumerate}

\subsection{Partes}
La regla dice $I=uv-\int vdu$
\begin{enumerate}
	\item Escoger u
	\begin{itemize}
		\item \textbf{I}nversas trigonométricas
		\item \textbf{L}ogaritmos
		\item \textbf{A}algebraicas
		\item \textbf{T}rigonométricas
		\item \textbf{E}xponenciales
	\end{itemize}
	\item Escoger dv, que es lo que queda al escoger u.
	\item Derivar u, no olvidar el dx
	\item Integrar dv
	\item Aplicar regla
	\item Resolver integral que queda por cualquier método.
\end{enumerate}

\subsubsection{Tablas}

$$\int x^ne^{algo lineal}dx\qquad\int x^ncos(algo lineal)dx\qquad \int x^nsen(algo lineal)dx$$

\subsection{Cíclicas}

$$\int e^xcos(algo lineal)dx\qquad \int e^xsen(algo lineal)dx\qquad \int e^x\#^xdx$$
\subsection{Trigonométricas}
$$\int sen^n(x)cos^m(x)dx$$
\begin{itemize}
	\item Si la potencia del seno es impar
	\begin{enumerate}
		\item Guardar un seno.
		\item Los senos restantes cambiarlos por coseno, $sen^2(x)=1-cos^2(x)$.
		\item Usar sustitución con $u=cos(x)$.
	\end{enumerate}
	\item Si la potencia del coseno es impar
	\begin{enumerate}
		\item Guardar un coseno.
		\item Los cosenos restantes cambiarlos por seno, $cos^2(x)=1-sen^2(x)$.
		\item Usar sustitución con $u=sen(x)$.
	\end{enumerate}
	\item Si ambas potencias son pares
	\begin{enumerate}
		\item Cambiar seno y coseno por $sen^2(x)=\dfrac{1-cos(2x)}{2}$ y $cos^2(x)=\dfrac{1+cos(2x)}{2}$
		\item Los cosenos restantes cambiarlos por seno, $cos^2(x)=1-sen^2(x)$.
		\item Usar sustitución con $u=sen(x)$.
	\end{enumerate}
	\item Recordar que $sen(x)cos(x)=\dfrac{1}{2}sen(2x)$
\end{itemize}

$$\int sec^n(x)tan^m(x)dx$$
\begin{itemize}
	\item Si la potencia del secante es par
	\begin{enumerate}
		\item Guardar un $sec^2(x)$.
		\item Los secantes restantes cambiarlos por tangente, $sec^2(x)=1+tan^2(x)$.
		\item Usar sustitución con $u=tan(x)$.
	\end{enumerate}
	\item Si la potencia del tangente es impar
	\begin{enumerate}
		\item Guardar un $sec(x)tan(x)$.
		\item Los tangente restantes cambiarlos por secante, $tan^2(x)=sec^2(x)-1$.
		\item Usar sustitución con $u=sec(x)$.
	\end{enumerate}
	
\end{itemize}

\subsection{Sustitución Trigonométrica}
\begin{enumerate}
	\item Raiz cuadrada
	\item Otro exponente
	\item Completación de cuadrados
	\item Número acompañando a $x^2$
\end{enumerate}


\subsection{Fracciones Parciales}
\begin{itemize}
	\item Si el grado del numerador es mayor que el grado del denominador, hacer división larga o división de polinomios y luego a aplicar fracciones parciales a la fracción que queda.
	\item Fracciones parciales
	\begin{enumerate}
		\item Factorizar el denominador
		\item Abrir en fracciones parciales
		\begin{itemize}
			\item Factores lineales, no tienen exponentes.
			\item Factores repetidos.
			\item Factores no lineales.
		\end{itemize}
		\item Encontrar constantes
		\item Integrar cada fracción.
	\end{enumerate}
\end{itemize}

\subsection{Sucesiones}
\begin{itemize}
	\item Una sucesión converge si $\lim\limits_{x -> \infty}a_n =\#$	
	\item Una sucesión diverge si $\lim\limits_{x -> \infty}a_n = \pm\infty$	
	\item Una sucesión es decreciente si $a_{n+1}<a_n$
	\item Una sucesión creciente si $a_{n+1}>a_n$
\end{itemize}


\textbf{Ejercicios}
\begin{enumerate}
	\item $a_n=ln(n+1)-ln(n)$
	\item Diga si converge y encuentre el límite.\\	
	$a_1=1$\\	
	$a_n=1+\dfrac{1}{1+a_{n-1}}$
	\item Demuestre que la sucesión definida por:\\
	$a_1=2$\\ 
	$a_{n+a}=\dfrac{1}{3-a_n}$\\
	\textit{Hint:} Se asume que todos los $a_n$, $a_{n+1}$, $a_{n-1}$... Convergen a L y se saca límite a ambos lados.
\end{enumerate}
\subsection{Series }
\subsubsection{Serie Geométricas}
Se usa cuando se tiene $\#^n$
 
\begin{itemize}
	\item Reescribir para que se parezca a $\sum_{n=0}^{\infty}ar^n$
	\item Si $|r|<1$ converge
	\item Si $|r|\ge1$ diverge
\end{itemize}

Si piden la suma o el valor al que converge:

\begin{itemize}
	\item $\sum_{n=0}^{\infty}ar^n\rightarrow s=\dfrac{a}{1-r}$
	\item $\sum_{n=1}^{\infty}ar^n\rightarrow s=a\left( \dfrac{1}{1-r}-r^0 \right) $
	\item $\sum_{n=2}^{\infty}ar^n\rightarrow s=a\left( \dfrac{1}{1-r}-r^0-r^1 \right) $
\end{itemize}
\subsubsection{Criterio de la divergencia}
Se usa con "Todos".

\begin{itemize}
	\item Hacer $\lim\limits_{n -> \infty} a_n$
	\item Si de un $\#$ diverge
	\item Si da 0 no concluye
\end{itemize}
\subsubsection{Series Telescópicas}
Solo se usa si se tiene "Resta de raíces", "Logaritmos", $\dfrac{polinomio}{polinomio}$  y el denominador se puede factorizar y solo se usa si dicen calcule su suma o diga a donde converge.\\
\textbf{Pasos}
\begin{enumerate}
	\item Factorizar el denominador
	\item Fracciones parciales
	\item abrir la serie\\
	Derecha $n=min, min+1, min+2\dots$ hasta que se cancelen 2 términos seguidos.\\
	Izquierda $n=n, n-1, n-2\dots$ hasta que se cancelen 2 términos seguidos.
	\item $s_n$ es lo que no se cancela.
	\item $\lim\limits_{n -> \infty} s_n$ Si da un $\#$ converge y converge a ese $\#$. Si da infinito diverge.
\end{enumerate}
\subsubsection{Criterio de la integral}
Se usa como última opción, cuando se tienen $\ln$ o una integral fácil (sustitución o partes sencillas).

\textbf{Pasos}
\begin{enumerate}
	\item Continua. Encontrar el dominio y concluir.
	\item Positiva. Mirar el tipo de función que se tiene y concluir.
	\item Decreciente. Derivar y hacer tabla de signos con 2 números o hacer $a_{n+1}<a_n$.
	\item Resolver la integral. Usar integrales impropias.
	\item Si la integral da un $\#,\ \sum_{n=0}^{\infty}$ converge. Si da $\pm\infty$ diverge.
\end{enumerate}
\subsubsection{P-series}
Se usa cuando se tiene $\dfrac{1}{n^p}$ y p es un $\#$. 
\begin{itemize}
	\item Si $|p|>1$ converge
	\item Si $|p|\le1$ diverge
\end{itemize}
\subsubsection{Criterio de comparación}
Se usa para trigonométricas o logaritmos\\
\textit{Hint:} $$-1\leq sen(algo)\leq 1$$ $$-1\leq cos(algo)\leq 1$$ $$-\dfrac{\pi}{2}\leq tan^{-1}(algo)\leq \dfrac{\pi}{2}$$ $$\ln(n)\leq n$$ $$\dfrac{1}{n^2+algo}<\dfrac{1}{n^2}$$ $$\dfrac{1}{\#^n+algo}<\dfrac{1}{\#^n}$$ $$\dfrac{1}{n-algo}>\dfrac{1}{n}$$

\begin{itemize}
	\item Si $a_n\le b_n$ y $\sum b_n$ converge, entonces $\sum a_n$ converge
	\item Si $a_n\ge b_n$ y $\sum b_n$ diverge, entonces $\sum a_n$ diverge
\end{itemize}

\subsubsection{Criterio de comparación en el límite}

\textbf{Pasos}
\begin{enumerate}
	\item Encontrar $b_n=\dfrac{\text{Término mayor grado num}}{\text{Término mayor grado den}}$
	\item Ver si $\sum b_n$ converge o diverge.
	\item Si $\lim\limits_{n -> \infty}\dfrac{a_n}{b_n}=\#$ entonces si $\sum b_n$ converge $\sum a_n$ converge, pero si $\sum b_n$ diverge entonces $\sum a_n$ diverge. (Si hay raíz o exponente no hacer l'hopital sino dividir por la mayor potencia del denominador).
\end{enumerate}

\subsubsection{Series Alternantes}
Tiene $(-1)^n$, estas también se podrían hacer por criterio de la razón.

\begin{itemize}
	\item Hacer la serie sin $(-1)^n$ por cualquier otro método. Si converge, $\sum a_n$  converge absolutamente. Si diverge, pasar al paso 2
	\item Serie alternante. Hacer $\lim\limits_{n -> \infty}a_n$ sin $(-1)^n$
	\item Si da 0 mostrar que es decreciente (Derivada) y converge condicionalmente.
	\item Si da un $\#$ diverge por criterio de divergencia 
	
\end{itemize}

\subsubsection{Criterio de la razón o cociente}
$n!$ o cualquiera

\begin{itemize}
	\item Hacer $\lim\limits_{n -> \infty}\left| \dfrac{a_{n+1}}{a_n} \right| $, si
	\begin{align*}
	\lim\limits_{n -> \infty}\left| \dfrac{a_{n+1}}{a_n} \right|<&1,\qquad \text{Converge}\\
	\lim\limits_{n -> \infty}\left| \dfrac{a_{n+1}}{a_n} \right|>&1,\qquad \text{Diverge}\\
	\lim\limits_{n -> \infty}\left| \dfrac{a_{n+1}}{a_n} \right|=&1,\qquad \text{Usar otro método}\\
	\end{align*}
\end{itemize}

\subsubsection{Criterio de la raíz }
$(Todo)^n$

Se hace $\lim\limits_{n -> \infty}\sqrt[n]{|a_n|}$, si


\subsection{Series de Potencias}
Cuando hay x, preguntan intervalo o radio de convergencia

\textbf{Pasos}
\begin{enumerate}
	\item Aplicar criterio de la razón
	\item Sacar constantes del limite con valor absoluto (lo que tenga x o sea constante).
	\item Evaluar el límite
	\begin{itemize}
		\item Si el límite da cero, $R=\infty$, intervalo de convergencia $(-\infty,\infty)$
		\item Si el límite da $\infty$, $R=0$, Intervalo de convergencia es un $\#$.
		\item Si el límite da un $\#$, pasar al paso 4.
	\end{itemize}
	\item Poner lo que dió el paso 3 $< 1$ "Siempre".
	\item Abrir valor absoluto $|algo|<\# \rightarrow -\#<algo<\#$
	\item Resolver valor absoluto $a<x<b$ y encontrar el radio$=\dfrac{b-a}{2}$
	\item Reemplazar a y b por separado en la x de la serie original y evaluar si las series resultantes convergen o no. En caso de que converja se incluye el valor del que converja en el intervalo.
\end{enumerate}

\subsection{Representación en Series de Potencia}
\textbf{Representaciones conocidas}
\begin{enumerate}
	\item $\dfrac{1}{1-x}=\sum_{n=0}^{\infty}x^n \rightarrow \dfrac{1}{1-algo}=\sum_{n=0}^{\infty}(algo)^n$
	\item $e^x=\sum_{n=0}^{\infty}\dfrac{x^n}{n!} \rightarrow e^{algo}=\sum_{n=0}^{\infty}\dfrac{algo^n}{n!}$
	\item $\sin(x)=\sum_{n=0}^{\infty}\dfrac{(-1)^nx^{2n+1}}{(2n+1)!} \rightarrow \sin(algo)=\sum_{n=0}^{\infty}\dfrac{(-1)^n(algo)^{2n+1}}{(2n+1)!}$
	\item $\cos(x)=\sum_{n=0}^{\infty}\dfrac{(-1)^nx^{2n}}{(2n)!} \rightarrow \cos(algo)=\sum_{n=0}^{\infty}\dfrac{(-1)^n(algo)^{2n}}{(2n)!}$
	\item $\tan^{-1}(x)=\sum_{n=0}^{\infty}\dfrac{(-1)^nx^{2n+1}}{(2n+1)} \rightarrow \tan^{-1}(algo)=\sum_{n=0}^{\infty}\dfrac{(-1)^n(algo)^{2n+1}}{(2n+1)}$
\end{enumerate}

Se debe tratar de reescribir la que nos dan como una de las anteriores.\\
\textbf{Ejemplo}
$f(x)=\dfrac{x}{9+x^2}$

\textbf{Si se tiene ln(algo)}
\begin{enumerate}
	\item Se deriva la función
	\item Representar como serie de potencias.
	\item Integrar la serie (con respecto a x, lo que tiene x).
\end{enumerate}

\subsection{Series de Taylor}
\textbf{Serie de Taylor}: $a\not=0$
$$f(x)=\sum_{n=0}^{\infty}\dfrac{f^{(n)}(a)(x-a)^n}{n!}$$

\textbf{Serie de Maclaurin}: $a=0$
$$f(x)=\sum_{n=0}^{\infty}\dfrac{f^{(n)}(0)(x)^n}{n!}$$

\textbf{Pasos}
\begin{enumerate}
	\item Derivar hasta encontrar un patrón (Derivada n-ésima).
	\item Reemplazar el a en x.
	\item Reemplazar en la fórmula. 
\end{enumerate}

\textbf{Ejercicio}\\
Sea $f(x)=ln(4-3x^2)$ encontrar la serie de Taylor centrada en x=0 y calcular $f^{(10)}=0$

\begin{enumerate}
	\item $f'(x)=\dfrac{1}{4-3x^2}(-6x)=-6x\dfrac{1}{4-3x^2}$
	\item $-6x\dfrac{\dfrac{1}{4}}{\dfrac{4}{4}-\dfrac{3x^2}{4}}=\dfrac{-6x}{4}\dfrac{1}{1-\dfrac{3x^2}{4}}=\dfrac{-6x}{4}\sum_{n=0}^{\infty}\dfrac{3^nx^{2n}}{4^n}=\dfrac{-6}{4}\sum_{n=0}^{\infty}\dfrac{3^nx^{2n+1}}{4^n}$
	\item $ln(4-3x^2)=\dfrac{-6}{4}\sum_{n=0}^{\infty}\dfrac{3^nx^{2n+2}}{(2n+2)4^n}$
	\item b. $\dfrac{-6}{4}\sum_{n=0}^{\infty}\dfrac{3^nx^{2n+2}}{(2n+2)4^n}=f(x)=\sum_{n=0}^{\infty}\dfrac{f^{(n)}(0)(x)^n}{n!}$\\
	$\dfrac{-6}{4}\dfrac{3^nn!}{(2n+2)4^n}=f^{(n)}(0)$\\
	$\dfrac{-6}{4}\dfrac{3^{10}10!}{(2*10+2)4^{10}}=f^{(10)}(0)$
\end{enumerate}

\subsection{Aproximación de Taylor}
\textbf{Aproximación lineal}

$$L(x)=f(a)+f'(a)(x-a)$$

\textbf{Aproximación con polinomio de grado n}

$$P_n(x)=f(a)+f'(a)(x-a)+\dfrac{f''(a)(x-a)^2}{2}+\dots+\dfrac{f^{(n)}(a)(x-a)^n}{n!}$$


\subsection{Excedentes de consumidor y productor}
Triángulo superior, lo que se deja de consumir, lo que están dispuestos a pagar de mas por una cantidad menor
$$EC=\int_{0}^{q^*}Demanda\ dq-p^*q^*\qquad \text{,p despejado}$$
$$EC=\int_{p^*}^{b}Demanda\ dp \qquad \text{,q despejado}$$
Triángulo inferior, lo que se deja de consumir, lo que están dispuestos a rebajar por una cantidad menor
$$EP=p^*q^*-\int_{0}^{q^*}Oferta\ dq\qquad \text{,p despejado}$$
$$EP=\int_{a}^{p^*}Oferta\ dp\qquad \text{,q despejado}$$

\subsection{Índice de Gini}
Lorentz: L(x) es como se distribuye la riqueza en la población
$$Gini=2\int_{0}^{1}x-L(x)=1-2\int_{0}^{1}L(x)$$
Entre mas pequeño sea el indice, menor es la desigualdad.





\newpage
%
%\centering{\textbf{Quiz}
%\begin{enumerate}
%	\item Para las siguientes series determine si converge o diverge. Justifique
%	\begin{enumerate}
%		\item $\displaystyle\sum_{n=1}^{\infty}\dfrac{1}{3+5^n}$
%		\item $\displaystyle\sum_{n=1}^{\infty}\dfrac{3(n!)^2}{10(2n)!}$
%		\item $\displaystyle\sum_{n=1}^{\infty}(-1)^n\dfrac{n^n}{n!}$
%		\item $\displaystyle\sum_{k=2}^{\infty}\dfrac{1}{k\sqrt{\ln k}}$
%		\item $\displaystyle\sum_{n=1}^{\infty}\dfrac{\sin(2n)}{1+2^n}$
%	\end{enumerate}
%	
%	\item Encuentre el radio e intervalo de convergencia para la siguiente serie
%	
%	$$\displaystyle\sum_{k=0}^{\infty}\dfrac{(-2)^k(x+3)^k}{3^{k+1}}$$
%\end{enumerate}

