\section{ÁLGEBRA}

\subsection{Vectores}
Un vector es la linea que une dos puntos, la cual tiene magnitud (longitud) y dirección. Está compuesto por componentes.

\textbf{Definición geométrica:} El conjunto de todos los segmentos de recta dirigidos equivalentes a un segmento de
recta dirigido dado se llama vector
\subsubsection{Suma y resta}
Coordenada a coordenada
\subsubsection{Multiplicación por un escalar}
Se multiplica cada coordenada

\subsubsection{Norma de un vector}
$||\overrightarrow{a}||=\sqrt{a_1^2+a_2^2+a_3^2}$

\subsubsection{Dirección de un vector}
$\theta=arctan(\frac{b}{a})$

\subsubsection{Vector unitario}
$u=\frac{\overrightarrow{v}}{||\overrightarrow{v}||}$

\subsubsection{Producto punto}
Se multiplican las coordenadas y se suman. Da un $\#$.

$\overrightarrow{u}\cdot\overrightarrow{v}=||\overrightarrow{u}||||\overrightarrow{v}||cos(\theta)$

\subsubsection{Ángulo entre 2 vectores}
$cos\theta=\dfrac{\overrightarrow{a}\cdot \overrightarrow{b}}{||\overrightarrow{a}||||\overrightarrow{b}||}$

\subsection{Proyecciones}
\begin{itemize}
	\item Proyección de u sobre v
	$proy_{\overrightarrow{v}}\overrightarrow{u}=\dfrac{\overrightarrow{u}\cdot \overrightarrow{v}}{\overrightarrow{v}\cdot \overrightarrow{v}}\overrightarrow{v}=\dfrac{\overrightarrow{u}\cdot \overrightarrow{v}}{|\overrightarrow{v}|^2}\overrightarrow{v}$.
	\item La $proy_{\overrightarrow{v}}\overrightarrow{u}$ es paralelo a $\overrightarrow{v}$.
	\item $u-proy_{\overrightarrow{v}}\overrightarrow{u}$ es ortogonal a $\overrightarrow{v}$.
\end{itemize}

\subsubsection{Distancia entre 2 puntos}
\begin{enumerate}
	\item Encontrar el vector de A a B $\overrightarrow{AB}=B-A$
	\item La distancia es la norma $||\overrightarrow{AB}||$
\end{enumerate}

\subsubsection{Cosenos directores}
Los ángulos directores del vector son: $\overrightarrow{v}=(x,y,z)$

\begin{itemize}
	\item $cos\alpha=\dfrac{x}{||\overrightarrow{v}||}$
	\item $cos\beta=\dfrac{y}{||\overrightarrow{v}||}$
	\item $cos\gamma=\dfrac{z}{||\overrightarrow{v}||}$
	\item $cos^2\alpha+cos^2\beta+cos^2\gamma=1$
\end{itemize}


\subsubsection{Producto cruz}
El producto cruz da como resultado un vector ortogonal o perpendicular a los vectores con los que se calcula. Para 2 vectores $\overrightarrow{u}=(u_1,u_2,u_3)$ y $\overrightarrow{v}=(v_1,v_2,v_3)$.

$\overrightarrow{u} \times \overrightarrow{v}=
\left|\begin{tabular}{ccc}
$\hat{i}$ & $\hat{j}$ & $\hat{k}$\\
$u_1$ & $u_2$ & $u_3$\\
$v_1$ & $v_2$ & $v_3$
\end{tabular}\right|=(u_2v_3-u_3v_2,u_3v_1-u_1v_3,u_1v_2-u_2v_1)
$

\begin{itemize}
	\item $\overrightarrow{v} \times \overrightarrow{u}=-\overrightarrow{u} \times \overrightarrow{v}$
	\item $||\overrightarrow{u} \times \overrightarrow{v}||= ||\overrightarrow{u}||||\overrightarrow{v}||sen(\theta)$
	\item El área del paralelogramo que tiene lados $\overrightarrow{u}$ y $\overrightarrow{u}$  es $||\overrightarrow{u} \times \overrightarrow{v}||$
	\item El volumen del parelepipedo formado por $\overrightarrow{u}$, $\overrightarrow{v}$ y $\overrightarrow{w}$ es $|(\overrightarrow{u} \times \overrightarrow{v})\cdot w|$
\end{itemize}




\subsubsection{Propiedades}
\begin{itemize}
	\item Dos vectores son paralelos si $\overrightarrow{v}=\alpha\overrightarrow{u}$
	\item Dos vectores son paralelos si $\overrightarrow{u}\times \overrightarrow{v}=0$
	\item Dos vectores son perpendiculares si $\overrightarrow{u}\cdot\overrightarrow{v}=0$
\end{itemize}



\subsection{Rectas y planos}
\subsubsection{Ecuación paramétrica de recta}
Pasa por el punto $a=(a_1,a_2,a_3)$ y tiene como vector director $\overrightarrow{d}=(d_1,d_2,d_3)$
\begin{align*}
	X=& a_1+d_1t\\
	Y=& a_2+d_2t\\
	Z=& a_3+d_3t\\
\end{align*}
\subsubsection{Ecuación simétrica de recta}
$$\dfrac{x-a_1}{d_1}=\dfrac{y-a_2}{d_2}=\dfrac{z-a_3}{d_3}$$

\subsubsection{Ecuación del plano}
Pasa por el punto $P=(x_0,y_0,z_0)$ y tiene como vector normal o perpendicular al plano $\overrightarrow{n}=(n_1,n_2,n_3)$. Tener en cuenta que $n\cdot (vector\ plano)=0$
$$n_1(x-x_0)+n_2(y-y_0)+n_3(z-z_0)=0$$

\begin{itemize}
	\item Dos planos son paralelos si sus vectores normales con paralelos
	\item Dos planos son perpendiculares si sus vectores normales son perpendiculares.
\end{itemize}
\subsubsection{Tipos de pregunta}
\begin{itemize}
	\item Para ver si un punto está en la recta.
	\begin{enumerate}
		\item Reemplazar el punto en x, y y z.
		\item Despejar t de cada ecuación
		\item Conclusión. Si todas las t dan iguales el punto está en la recta. Si no dan iguales no está en la recta.
	\end{enumerate}
	\item Para encontrar la intersección entre 2 rectas
	\begin{enumerate}
		\item Igualar las ecuaciones
		\item Despejar t de cada ecuación
		\item Si la t se cancela y da 0=0, hay infinitas soluciones.Si da $\#=0$, no se interceptan.
		\item Si todas las t dan iguales si hay intersección. Si no dan iguales no hay intersección.
		\item Reemplazar t en cualquier ecuación para encontrar punto de intersección.
	\end{enumerate}
	\item Para encontrar la intersección entre recta y plano
	\begin{enumerate}
		\item Reemplazar la recta en el plano
		\item Despejar t
		\item Si la t se cancela y da 0=0, hay infinitas soluciones.Si da $\#=0$, no se interceptan.
		\item Si la t no se cancela reemplazar en la recta para encontrar el punto de intersección
	\end{enumerate}
	\item Para encontrar un plano si dan una recta perpendicular.
	\begin{enumerate}
		\item $\overrightarrow{n}=\overrightarrow{d}$
		\item Reemplazar en ecuación del plano.
	\end{enumerate}
	\item Para encontrar una recta si dan un plano perpendicular.
	\begin{enumerate}
		\item $\overrightarrow{d}=\overrightarrow{n}$
		\item Reemplazar en las ecuaciones paramétricas
	\end{enumerate}
	
	\item Para encontrar un plano si dan 3 puntos.
	\begin{enumerate}
		\item Fijar un punto (a)
		\item Hallar 2 vectores con respecto a ese punto (ab y ac). Restar a b y a c el punto a
		\item Hacer producto cruz para encontrar vector normal.
		\item Reemplazar en la ecuación del plano el punto y el vector normal.
	\end{enumerate}
	\item Para encontrar la intersección entre 2 planos.
	\begin{enumerate}
		\item Volver las ecuaciones una matriz ampliada.
		\item Hacer Gauss.
	\end{enumerate}
	\item Si dan 2 vectores en el plano y un punto y piden ecuación de plano.
	\begin{enumerate}
		\item Hacer producto cruz para encontrar vector normal
		\item Reemplazar en la ecuación del plano el punto y el vector normal.
	\end{enumerate}
	\item Distancia entre dos planos paralelos
	\begin{enumerate}
		\item Encontrar un punto en cada plano
		\item Armar vector con esos puntos
		\item Encontrar la proyección sobre uno de los vectores normales
		\item La norma de la proyección es la distancia entre los planos.
		\item También se puede usar la fórmula $D=\dfrac{|Ax_0+By_0+Cz_0+D|}{\sqrt{A^2+B^2+C^2}}$, donde D está al lado izquierdo del igual.
	\end{enumerate}
\end{itemize}

\subsection{Matrices}
Se escriben con letras mayúsculas. $A_{\underbrace{n}_{{filas}}\times \underbrace{m}_{columnas}}
$
\subsubsection{Suma y resta}
Se hace coordenada a coordenada, las matrices deben tener el mismo tamaño
\subsubsection{Multiplicación por un escalar}
Se multiplica cada coordenada.
\subsubsection{Multiplicación entre matrices}
$A_{n\times m}B_{p\times q}=C_{n\times q}$. Para que se pueda hacer la multiplicación $m=q$

\subsection{Eliminación de Gauss Jordan y Eliminación Gaussiana}

\subsubsection{Operaciones}
\begin{itemize}
	\item Multiplicar (o dividir) un renglón por un número diferente de cero. $R_1\rightarrow aR1$
	\item Sumar un múltiplo de un renglón a otro renglón. $R_2\rightarrow R2+aR1$
	\item Intercambiar dos renglones. $R_1\leftrightarrow3$
\end{itemize}

\subsubsection{Forma escalonada reducida por renglones (Eliminación de Gauss Jordan)}

$\begin{pmatrix}
1_{(1)} & 0_{(4)} & 0_{(6)}\\
0_{(2)} & 1_{(3)} & 0_{(6)}\\
0_{(2)} & 0_{(4)} & 1_{(5)}\\
\end{pmatrix}$
\\

Puede haber una fila de ceros en la parte inferior
\subsubsection{Forma escalonada por renglones (Eliminación Gaussiana)}

$\begin{pmatrix}
1_{(1)} & \# & \#\\
0_{(2)} & 1_{(3)} & \#\\
0_{(2)} & 0_{(4)} & 1_{(5)}\\
\end{pmatrix}$
\\

Puede haber una fila de ceros en la parte inferior
\subsubsection{Tipos de solución}
\begin{itemize}
	\item Infinitas soluciones
	
	$\begin{pmatrix}
	1 & \# & \#&|&\#\\
	0 & 1 & \#&|&\#\\
	0 & 0 & \underbrace{\#}_{=0}&|&\underbrace{\#}_{=0}\\
	\end{pmatrix}$
	\item No tiene solución
	
	$\begin{pmatrix}
	1 & \# & \#&|&\#\\
	0 & 1 & \#&|&\#\\
	0 & 0 & \underbrace{\#}_{=0}&|&\underbrace{\#}_{\not=0}\\
	\end{pmatrix}$
	\item Única solución
	
	$\begin{pmatrix}
	1 & \# & \#&|&\#\\
	0 & 1 & \#&|&\#\\
	0 & 0 & \underbrace{\#}_{\not=0}&|&\underbrace{\#}_{No importa}\\
	\end{pmatrix}$
\end{itemize}


Se llama solución trivial cuando todas las variables valen cero.
\subsection{Matriz inversa $A^{-1}$}

\subsection{2x2}
Si $A=\begin{pmatrix}
a & b \\
c & d 
\end{pmatrix}$, entonces \\
$A^{-1}=\dfrac{1}{\det(A)}\begin{pmatrix}
d & -b \\
-c & a 
\end{pmatrix}$

\subsubsection{Gauus}
$$(A|I)\rightarrow^{gauss}(I|A^{-1})$$
\subsubsection{Adjunta}
$$A^{-1}=\dfrac{1}{|A|}adj(A)$$
$$adj(A)=[cofactores]^T$$

\subsubsection{Propiedades}
\begin{itemize}
	\item $AA^{-1}=I$
	\item $(AB)^{-1}=B^{-1}A^{-1}$
	\item $(\#A)^{-1}=\dfrac{1}{\#}A^{-1}$
	\item $(A^T)^{-1}=(A^{-1})^T$
	\item $(A^{-1})^{-1}=A$
	\item Si A es invertible, el sistema tiene $A\overrightarrow{x}=\overrightarrow{b}$ se puede solucionar como $\overrightarrow{x}=A^{-1}\overrightarrow{b}$
	\item Una matriz es invertible si $|A|\not=0$
	\item Una matriz cuadrada que no tiene inversa se denomina singular y si es invertibles se denomina no singular.
\end{itemize}

\subsection{Transpuesta}
$A^T:$ cambiar filas por columnas.
\subsubsection{Propiedades}
\begin{itemize}
	\item Una matriz es simétrica si $A^T=A$
	\item Una matriz es antisimétrica si $A^T=-A$
	\item Una matriz es ortogonal si $A^TA=I$
	\item $(A^T)^T=A$
	\item $(AB)^T=B^TA^T$
	\item $(\alpha A)T=\alpha A^T$
	\item $(A+B)^T=A^T+B^T$
	\item $(A^T)^{-1}=(A^{-1})^T6$
\end{itemize}

\subsection{Matrices elementales}
Son matrices que se pueden obtener a partir de la matriz identidad, mediante una sola operación elemental (Multiplicación $cR_i$, suma de filas $R_j+cR_i$ e intercambio de filas $P_{ij}$)

\begin{itemize}
	\item Para realizar una operación a la Matriz A, se multiplica por la izquierda la matriz elemental.
	\item Inversas de las elementales $(cR_i)^{-1}=\dfrac{1}{c}R_i$ $(R_j+cR_i)^{-1}=R_j-cR_i$ y $(P_{ij})^{-1}=P_{ij}$
	\item $I=E_mE_{m-1}\dots E_{2}E_1A$ 
	\item $A^{-1}=E_mE_{m-1}\dots E_{2}E_1$ y $A=E_1^{-1}E_{2}^{-1}\dots E_2^{m-1}E_1^{m}$
\end{itemize}

\subsection{Determinantes}
\subsubsection{Matriz 2x2}

$
\left|\begin{tabular}{ccc}
	a & b \\
	c & d
\end{tabular}\right|=ad-bc
$

\subsubsection{Sarrus}
\begin{tikzpicture}
\matrix [%
matrix of math nodes,
column sep=1em,
row sep=1em
] (sarrus) {%
	a_{11} & a_{12} & a_{13} & a_{11} & a_{12} \\
	a_{21} & a_{22} & a_{23} & a_{21} & a_{22} \\
	a_{31} & a_{32} & a_{33} & a_{31} & a_{32} \\
}; 

\path ($(sarrus-1-3.north east)+(0.5em,0)$) edge[dotted] ($(sarrus-3-3.south east)+(0.5em,0)$)
(sarrus-1-1)                          edge         (sarrus-2-2)
(sarrus-2-2)                          edge         (sarrus-3-3)
(sarrus-1-2)                          edge         (sarrus-2-3)
(sarrus-2-3)                          edge         (sarrus-3-4)
(sarrus-1-3)                          edge         (sarrus-2-4)
(sarrus-2-4)                          edge         (sarrus-3-5)
(sarrus-3-1)                          edge[dashed] (sarrus-2-2)
(sarrus-2-2)                          edge[dashed] (sarrus-1-3)
(sarrus-3-2)                          edge[dashed] (sarrus-2-3)
(sarrus-2-3)                          edge[dashed] (sarrus-1-4)
(sarrus-3-3)                          edge[dashed] (sarrus-2-4)
(sarrus-2-4)                          edge[dashed] (sarrus-1-5);

\foreach \c in {1,2,3} {\node[anchor=south] at (sarrus-1-\c.north) {$+$};};
\foreach \c in {1,2,3} {\node[anchor=north] at (sarrus-3-\c.south) {$-$};};
\end{tikzpicture}

\subsubsection{Matriz nxn, cofactores}
\begin{enumerate}
	\item Escoger fila o columna con mas ceros
	\item Poner signos imaginarios
	\item Tapar y calcular el determinante de lo que queda.
\end{enumerate}

\subsubsection{Propiedades}
\begin{itemize}
	\item $|A|^T=|A|$
	\item $|A^{-1}|=\dfrac{1}{|A|}$
	\item $|\#A|=\#^n|A|$
	\item $|AB|=|A||B|$
	\item $|A+B|\not = |A|+|B|$
	\item $|adjA|=|A|^{n-1}$
	\item Si hay una fila o columna de ceros $|A|=0$
	\item Si hay fila múltiplo de otra $|A|=0$
\end{itemize}

\subsection{Regla de Crammer}
Teniendo $A=(a\ b\ c)$ y el vector de respuestas R
\begin{enumerate}
	\item Encontrar determinante de la matriz, si es cero el sistema no tiene solución.
	\item $x=\dfrac{|R\  b\  c|}{|A|}$
	\item $y=\dfrac{|a\  R\  c|}{|A|}$
	\item $z=\dfrac{|a\  b\  R|}{|A|}$
\end{enumerate}

\subsection{Combinación lineal}

\begin{itemize}
	\item Escribir la combinación lineal $a\overrightarrow{u}+b\overrightarrow{v}+c\overrightarrow{w}=\overrightarrow{z}$
	\item Poner vectores en matriz hacia abajo y Hacer gauss.
	\item Lo que da son las contantes de la combinación lineal, a, b y c.
\end{itemize}

Después de hacer gauss
\begin{itemize}
	\item Rango: $\#$ de filas no nulas
	\item Nulidad: $\#$ de filas nulas. También lo llaman grados de libertad.
	\item Dimensión: Rango + nulidad
\end{itemize}

Explicar como se da la respuesta cuando una fila da ceros.

\subsection{Independecia lineal}

Un conjunto de vectores son L.I. si ningún vector se puede escribir como combinación de otros o si  $a\overrightarrow{u}+b\overrightarrow{v}+c\overrightarrow{w}=0$ tiene solo la solución trivial.

Un conjunto de vectores son L.I. si el determinante de la matriz conformada por los vectores es distinta de cero.

Si al hacer Gauss se llega a la identidad. Es lo mismo que ninguno se pueda escribir en términos de otro.



\subsection{Diagonalización}
\begin{enumerate}
	\item Hacer A-$\lambda I$, es restarle $\lambda$ a la diagonal.
	\item Calcular el determinante de la matriz anterior
	\item Igualar a cero el determinante y despejar $\lambda$ que son los autovalores.
	\item Reemplazar cada $\lambda$ en la matriz del paso 1 y hacer gauss para encontrar los autovectores.
	\item Encontrar matriz de autovectores P. Si piden que sea ortogonal, normalizar cada vector.
	\item Encontrar la inversa de P
	\item Hacer $D=P^{-1}AP$, debe dar una diagonal con los autovalores.
	\item $A^n=PD^nP^{-1}$
\end{enumerate}

\subsection{Espacios Vectoriales}
Conjunto de vectores que cumple los siguientes 10 axiomas
\begin{enumerate}
	\item Cerradura bajo la suma, si $x \text{ y } y  \in V$, $x+y \in V$
	\item Ley asociativa, si $ x, y, z \in V$, $(x+y)+z=x+(y+z)$
	\item Vector cero o idéntico aditivo $x+0=0+x=x$
	\item Inverso aditivo, si $x \in V$ $x+(-x)=0$
	\item Ley conmutativa de la suma, si $x \text{ y } y  \in V$ $x+y=y+x$
	\item Cerradura bajo la multiplicación por escalar, si $x \in V$  $\alpha x \in V$
	\item Ley distributiva 1, si $x \text{ y } y  \in V$  y $\alpha$ y $\beta$ son escalares $\alpha(x+y)=\alpha x+\alpha y$
	\item Ley distributiva 2,  $x \in V$  y $\alpha$ y $\beta$ son escalares $(\alpha + \beta)x=\alpha x+\beta x$
	\item Ley asociativa de la multiplicación por escalares, si $x \in V$ $\alpha(\beta x)=(\alpha \beta)x$
	\item Para cada vector $x \in V$, $1x=x$
\end{enumerate}

\subsection{Subespacios Vectoriales}
Un conjunto $H$ es un subespacio de $V$ si se cumple la cerradura sobre la suma y la cerradura de multiplicación por escalar y contiene al cero.

\subsection{Conjunto Generado}
Un conjunto de vectores $v_1,v_2,\dots,v_n$ generan a $V$ si todo vector en $V$ se puede escribir como una combinación lineal de $v_1,v_2,\dots,v_n$.

Si $v_1,v_2,\dots,v_n$ son vectores de un espacio vectorial V, entonces todas las combinaciones lineales de estos vectores forman un espacio generado y este a su vez es un subespacio del espacio V.

\subsection{Bases}
Un conjunto de vectores $v_1,v_2,\dots,v_n$ es una base para un espacio $V$ si 
\begin{itemize}
	\item $v_1,v_2,\dots,v_n$ son linealmente independientes
	\item $v_1,v_2,\dots,v_n$ generan a $V$
\end{itemize}

\begin{itemize}
	\item La dimensión de un espacio vectorial (dim V) es el número de vectores que haya en las bases.
	\item La nulidad es la dimensión del espacio nulo $N_A$. El espacio nulo es la solución  de $Ax=0$
	\item La base del espacio generado son los 1's que quedan al reducir.
	\item Todo vector en el espacio Fila $R_A$ de una matriz es ortogonal a todo vector en su espacio nulo.
	\item El rango es el número de pivotes en la forma escalonada.
	\item Rango+Nulidad=Columnas
	\item Imagen A=$C_A$
\end{itemize}

\subsubsection{Cambio de Base}

Matriz de transición de la Base 1 a la Base 2
\begin{enumerate}
	\item Plantear el sistema 
	$\begin{pmatrix}
	B2 | B1
	\end{pmatrix}$
	\item Reducir usando gauus jordan
	$\begin{pmatrix}
	I | A
	\end{pmatrix}$, $A$ es la matriz de transición de la Base 1 a la Base 2
	\item Para cambiar un vector de Base 1 a Base 2 $(X)_{B_2}=A(X)_{B_1}$
	\item $A^{-1}$ es la matriz de transición de la Base 2 a la Base 1
\end{enumerate}

\subsection{Transformaciones lineales}
Una Transformación $T:V\rightarrow W$ es lineal si
\begin{itemize}
	\item $T(u+v)=T_u+T_v$
	\item $T(\alpha v)=\alpha T_v$
\end{itemize}

Propiedades

\begin{itemize}
	\item El nucleo o kernel $nuT$ es la solución de $T_v=0$
	\item La imagen de T son las soluciones de $T_v=W$
	\item La nulidad de T es $dim\  nuT$
	\item El rango de T es $dim\ Imagen\ T$
	\item (Ejemplo 2, pag 500)Si se quiere que una transformación sea un plano se encuentra la base del plano, se transforman los vectores de la base y se transforma (x,y,z) en terminos de lo anterior. Es equivalente a despejar una letra y armar vector con sumas de x, y y z.
\end{itemize}

La matriz de transformación $A_T$
\begin{itemize}
	\item Si se tiene la base canónica
	\begin{itemize}
		\item Aplicar T a vectores de la base.
		\item Armar $A_T$ con las transformaciones como columnas.
	\end{itemize}
	\item (Pag 514) Transformar elementos de la Base 11, luego resolver $\begin{pmatrix}
	B2 | B1_T
	\end{pmatrix}$. La matriz que queda es la matriz de transformación
	\item (Pag 515) Si se tienen 2 bases no estándar $A_T=A_2^{-1}CA_1$ donde
	
	C: Matriz de coeficientes de la transformación.
	
	 $A_1$: matriz de transición de la base 1 a la base estándar.
	 
	 $A_2$: matriz de transición de la base 2 a la base estándar.
\end{itemize}

\subsection{Ortonormalización de Gram-Schmidt}
Sea $S={v_1,v_2,\dots,v_n}$ una base de H

\begin{enumerate}
	\item Hacer $u_1=v_1$
	\item $u_2=v_2-\dfrac{v_2\cdot u_1}{u_1\cdot u_1}u_1$
	\item $u_3=v_3-\dfrac{v_3\cdot u_1}{u_1\cdot u_1}u_1-\dfrac{v_3\cdot
	u_2}{u_2\cdot u_2}u_2$
	\item Normalizar $u_1,u_2,u_3$
\end{enumerate}

Una matriz es ortogonal si $Q^{-1}=Q^T$

\subsection{Proyección ortonormal}

La base debe ser ortonormal

$Proy_HV=(V\cdot u_1)u_1+(V\cdot u_2)u_2+(V\cdot u_3)u_3$
\subsection{Complemento ortogonal}

\begin{itemize}
	\item $H^{\perp}$: Complemento ortogonal, son los x tal que $x\cdot h=0, \ h\in H$
	\item El complemento ortogonal de W es el mismo espacio nulo de la matriz A cuyas filas son una base para el espacio W .
	\item $Dim\ H + Dim\ H^{\perp}=n$
	\item Si se tiene una  base, tomar la base como filas de una matriz y encontrar base para espacio nulo de esa matriz.
	\item $v=h+p=proy_Hv+proy_{H^{\perp}}v$
\end{itemize}






\section{R}
Funciones para exploración de datasets
\begin{itemize}
	\item head(): Imprime las primeras 6 posiciones
	\item tail(): Imprime las últimas 6 posiciones
	\item class(): Imprime la clase del objeto
	\item length(): 
	\item dim():
	\item summary(): Resumen dependiendo del tipo de parámetro
	\item str(): Da la estructura del objeto
\end{itemize}

Funciones comunes
\begin{itemize}
	\item sqrt():
	\item mean():
	\item sum():
	\item var():
	\item sd():
	\item exp(): 
	\item sin(): 
	\item cos(): 
	\item log(): 
\end{itemize}

Creación de vectores
\begin{itemize}
	\item c(): Sirve para crear un vector o para concatenar
	\item objeto1 = c(1,2,3,4)
	\item diez-cuatros = rep(4,10)
	\item cero-al-uno = seq(0,1, by = 0.1 )
	\item dos-al-ocho = 2:8
	\item diez-al-1 = 10:1
\end{itemize}

Valores Faltantes
\begin{itemize}
	\item is.na()
	\item 
\end{itemize}

Arreglos, matrices y listas
\begin{itemize}
	\item array(data=1:24,dim=c(2,4,3))
\end{itemize}

Variables Dummy

Objetos Data.Frame: Arreglo rectangular donde las columnas pueden ser de diferentes clases

Operador Pipe