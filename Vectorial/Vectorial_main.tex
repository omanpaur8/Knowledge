\section{VECTORIAL}

\subsection{Rectas}

$l(t)=a+\overrightarrow{v}t$
\subsection{Producto punto}
$\overrightarrow{u}\cdot\overrightarrow{v}=|\overrightarrow{u}||\overrightarrow{v}|cos(\theta)$

\subsection{Producto cruz}
$\overrightarrow{u}\times\overrightarrow{v}=|\overrightarrow{u}||\overrightarrow{v}|sen(\theta)$

La regla de la mano derecha indica la dirección del vector resultante del producto cruz.

\subsection{Proyectar una recta en un plano}
\begin{enumerate}
	\item Encontrar punto de intersección $p$
	\item Encontrar proyección de la recta sobre el vector normal $\overrightarrow{V_n}$
	\item Encontrar vector de proyección en el plano $\overrightarrow{V_H}=\overrightarrow{d}-\overrightarrow{V_n}$
	\item Armar ecuación de recta con punto $p$ y vector $\overrightarrow{V_H}$
\end{enumerate}
\subsection{Coordenadas cilíndricas}

$(r, \theta, z)$

\begin{itemize}
	\item $r=a$, cilindro
	\item $\theta=a$, plano
	\item $z=a$, plano horizontal
\end{itemize}

\subsubsection{Rectangulares a cilíndricas}
\begin{itemize}
	\item $r=\sqrt{x^2+y^2}$
	\item $\theta=arctan(\dfrac{y}{x})$
	\item $z=z$
\end{itemize}

\subsubsection{Cilíndricas a rectangulares}
\begin{itemize}
	\item $x=rcos(\theta)$
	\item $y=rsen(\theta)$
	\item $z=z$
\end{itemize}

\subsection{Coordenadas esféricas}
$(\rho, \phi,\theta)$

\begin{itemize}
	\item $\rho=a$, espera
	\item $\phi=a$, cono
	\item $\theta=a$, semiplano
\end{itemize}

\subsubsection{Rectangulares a esféricas}
\begin{itemize}
	\item $\rho=\sqrt{x^2+y^2+z^2}$
	\item $\phi=cos^{-1}(\dfrac{z}{\rho})$
	\item $\theta=tan^{-1}(\dfrac{y}{x})$
\end{itemize}

\subsubsection{Esféricas a rectangulares}
\begin{itemize}
	\item $x=\rho sin(\phi) cos(\theta)$
	\item $y=\rho sin(\phi) sin(\theta)$
	\item $z=\rho cos(\phi)$
\end{itemize}

\subsubsection{Cilíndricas a esféricas}
\begin{itemize}
	\item $\rho=\sqrt{r^2+z^2}$
	\item $\phi=sen^{-1}(\dfrac{r}{\rho})$
%	\item $\theta=tan^{-1}(\dfrac{y}{x})$
\end{itemize}


\subsection{Funciones}
\begin{itemize}
	\item Escalares: Al reemplazar dan un número
	\item Vectoriales: Al reemplazar dan un vector
\end{itemize}

\subsection{Representación geométrica}

\subsubsection{Curvas}

\begin{enumerate}
	\item Cambiar z o f(x,y) por k
	\item Despejar y (pero si hay $x^2+y^2$, despejar eso porque es un círculo)
	\item Darle valores a k, al menos 3
	\item Graficar para cada valor de k
	\item Graficar en 3D, para saber corte con z, reemplazar x y y por 0
\end{enumerate}

\subsubsection{Superficies de nivel}

\begin{enumerate}
	\item Cambiar f(x,y,z) por k
	\item Darle valores a k en distintos rangos para ver que superficies se forman.
	\item Graficar para cada valor de k
	\item Indicar que tipo de superficies se forman
\end{enumerate}

\subsubsection{Gráficas importantes}
\begin{itemize}
	\item Esfera: $x^2+y^2+z^2=\rho^2$, radio=$\rho$
	\item Cilindro:  $x^2+y^2=r^2$, la letra que no aparece es hacia donde abre
	\item Cono: $ z=\sqrt{x^2+y^2}$, la letra que no está en la raiz es hacia donde abre.
	\item Paraboloide: $ z=x^2+y^2$, el que no tiene el cuadrado es hacia donde abre.
	\item Elipsoide: $\dfrac{x^2}{a^2}+\dfrac{y^2}{b^2}+\dfrac{z^2}{c^2}=1$, a, b y c son los puntos donde corta en x, y y z respectivamente.
	\item Paraboloide hiperbólico o silla de montar: $ z=x^2-y^2$, el que no tiene el cuadrado es hacia donde abre.
\end{itemize}

\subsection{Límites y continuidad}
$\lim\limits_{(x,y) \to (x_0,y_0)}f(x,y)$

\begin{itemize}
	\item Los polinomios son continuos
	\item Utilizar varios acercamientos tratando de que de diferente por 2
	\item En caso de que el límite exista se debe mostrar por emparedado
\end{itemize}

\subsection{Derivadas Parciales}
\begin{itemize}
	\item Primeras derivadas: $f'_x=\dfrac{\delta f}{\delta x}$ $f'_y=\dfrac{\delta f}{\delta y}$
	\item Segundas derivadas: $f''_{xx}=\dfrac{\delta^2 f}{\delta x^2}$, $f''_{yy}=\dfrac{\delta^2 f}{\delta y^2}$, $f''_{xy}=\dfrac{\delta^2 f}{\delta x\delta y}$, $f''_{yx}=\dfrac{\delta^2 f}{\delta y\delta x}$
	\item Una función cuyas primeras derivadas parciales existan y sean continuas, se dice que es de clase $C^1$. Si una función es $C^1$ es diferenciable. Si estas derivadas a su vez tienen derivadas parciales continuas se die que F es de clase $C^2$ o que es 2 veces continuamente diferenciable y así sucesivamente.
	\item Si una función es clase $C^2$ entonces las derivadas parciales mixtas o cruzadas con iguales.
	\item $Df(x_0,y_0)$ es el vector o matriz de derivadas parciales (pag 125).
\end{itemize}

\subsubsection{Ecuación recta tangente}

\begin{itemize}
	\item Recta tangente a la intersección de dos superficies
	\begin{enumerate}
		\item Calcular el gradiente a las 2 superficies y reemplazar el punto dado
		\item Hacer producto cruz entre estos vectores y ese vector es el vector director de la recta.
		\item Armar la ecuación de la recta
	\end{enumerate}
\end{itemize}


\subsubsection{Ecuación plano tangente}

\begin{itemize}
	\item Si la función es de dos variables
	$$f'_x(x_0,y_0)(x-x_0)+f'_y(x_0,y_0)(y-y_0)=z-z_0$$
	\item Si la función es de 3 variables
	$$f'_x(x_0,y_0)(x-x_0)+f'_y(x_0,y_0)(y-y_0)+f'_z(x_0,y_0)(z-z_0)=0$$
	$$\nabla f(x_0,y_0,z_0)\cdot(x-x_0,y-y_0,z-z_0)=0$$	
\end{itemize}




\textbf{Pasos}

\begin{enumerate}
	\item Encontrar las primeras derivadas.
	\item Reemplazar el punto en las derivadas.
	\item Reemplazar en ecuación del plano.
\end{enumerate}

\subsubsection{Derivada Direccional}

$$\overrightarrow{D}_f=f'_x(x_0,y_0)h+f'_y(x_0,y_0)k$$
$$\overrightarrow{D}_f=\nabla f \cdot \overrightarrow{v}$$

Derivada direccional de f en al punto $(x_0,y_0)$ en la dirección del vector unitario (h,k).

\subsubsection{Regla de la cadena}
Si F(u,v) y u(x,y) v(x,y)
\begin{itemize}
	\item $f'_x=F'_uu'_x+F'_vv'_x$
	\item $f'_y=F'_uu'_y+F'_vv'_y$
\end{itemize}

Si h(t)=f(u,v,w) y (u(t), v(t) , w(t))
\begin{itemize}
	\item $\dfrac{dh}{dt}=\dfrac{\delta h}{\delta u}\dfrac{du}{dt}+\dfrac{\delta h}{\delta v}\dfrac{dv}{dt}+\dfrac{\delta h}{\delta w}\dfrac{dw}{dt}$
\end{itemize}

Si h(x,y,z)=f(u,v,w) y (u(x,y,z), v(x,y,z) , w(x,y,z))
\begin{itemize}
	\item $\dfrac{\partial h}{\partial x}=\dfrac{\partial h}{\partial u}\dfrac{\partial u}{\partial x}+\dfrac{\partial h}{\partial v}\dfrac{\partial v}{\partial x}+\dfrac{\partial h}{\partial w}\dfrac{\partial w}{\partial x}$
	\item $\dfrac{\partial h}{\partial y}=\dfrac{\partial h}{\partial u}\dfrac{\partial u}{\partial y}+\dfrac{\partial h}{\partial v}\dfrac{\partial v}{\partial y}+\dfrac{\partial h}{\partial w}\dfrac{\partial w}{\partial y}$
	\item $\dfrac{\partial h}{\partial z}=\dfrac{\partial h}{\partial u}\dfrac{\partial u}{\partial z}+\dfrac{\partial h}{\partial v}\dfrac{\partial v}{\partial z}+\dfrac{\partial h}{\partial w}\dfrac{\partial w}{\partial z}$
\end{itemize}

Si f(u,v) y g(x,y) entonces $Df\circ g=Df(g(x_0,y_0))Dg(x_0,y_0)$ pag 140.

\subsection{Teorema de Taylor (4.1)}
\subsubsection{Aproximación lineal o de primer orden}

$f(x,y)=f(a,b)+f'_x(a,b)(x-a)+f'_y(a,b)(y-b)$, son los mismos pasos del plano

\subsubsection{Aproximación de segundo orden}

$f(x,y)=f(a,b)+f'_x(a,b)(x-a)+f'_y(a,b)(y-b)+\dfrac{1}{2}f''_{xx}(a,b)(x-a)^2+\dfrac{1}{2}f''_{xy}(a,b)(y-b)^2$, Revisar en el libro (242)

\subsection{Máximos y mínimos sin región (4.2)}
\begin{enumerate}
	\item Encontrar derivadas parciales con respecto a x y y, $f'_{x}$ y $f'_{y}$.
	\item Igualar a cero y despejar las variables, esos son los puntos críticos o estacionarios.
	\item Encontrar las segundas derivadas $f''_{xx}$, $f''_{yy}$ y $f''_{xy}$
	\item Encontrar el discriminante $D$
	
	H=$\begin{pmatrix}
	f''_{xx} & f''_{xy} \\
	f''_{xy} & f''_{yy} \\
	\end{pmatrix}$
	
	$D=\det H$ o $D = f''_{xx}f''_{yy}-(f''_{xy})^2$
	\item Conclusión 
	\begin{itemize}
		\item Si $f''_{xx}>0$ y $D>0$ es un mínimo
		\item Si $f''_{xx}<0$ y $D>0$ es un máximo
		\item Si $D<0$ es un punto de silla
	\end{itemize}
\end{enumerate}

Ej: $f(x,y)=x^3+y^3-9xy+27$ en (0,0) (0,1) (1,1)

\subsection{Máximos y mínimos absolutos}

\begin{enumerate}
	\item Encontrar derivadas parciales con respecto a x y y, $f'_{x}$ y $f'_{y}$.
	\item Igualar a cero y despejar las variables, esos son los puntos críticos o estacionarios.
	\item Encontrar las segundas derivadas $f''_{xx}$, $f''_{yy}$ y $f''_{xy}$
	\item Dibujar región
	\item Nombrar los bordes de la región.
	\item Calcular para cada borde una función, reemplazando el borde en la original.
	\item Derivar e igualar a cero cada función y encontrar puntos
	\item Escoger puntos que están en la región y las esquinas (En un círculo las esquinas son los cortes con el eje x).
	\item Reemplazar todos los puntos en la original.
	\item Verificar cual es el máx y cual es el mín absoluto. 
\end{enumerate}

\subsection{Lagrange (4.3)}

$$\nabla f(x_1,x_2,\dots,x_n)=\lambda \nabla g(x_1,x_2,\dots,x_n)$$
$$g(x_1,x_2,\dots,x_n)=c$$

Si preguntan condiciones necesarias y suficientes. Las condiciones necesarias son las primeras derivadas $\mathcal{L}_x$, $\mathcal{L}_y$ y $\mathcal{L}_\lambda$ y las condiciones suficientes es que es que el punto que se encuentre sea máximo o mínimo de acuerdo a lo que se está pidiendo.

\begin{enumerate}
	\item Plantear Lagrangiano, $\mathcal{L}=funcion - \lambda(restriccion)$. La restricción debe estar igualada a cero.
	\item Encontrar $\mathcal{L}_x$, $\mathcal{L}_y$ y $\mathcal{L}_\lambda$
	\item Igualar a cero todas las derivadas y despejar $\lambda$ de cada una.
	\item Igualar los $\lambda$ y encontrar alguna variable.
	\item Reemplazar esa variable en la restricción y encontrar todas las variables.
	\item Verificar que cumple.
	\begin{itemize}
		\item Método 1: Hessiano limitado
		
		$$H=\begin{pmatrix}
		0 & -g'_x & -g'_y \\
		-g'_x & \mathcal{L}''_{xx} & \mathcal{L}''_{xy} \\
		-g'_y & \mathcal{L}''_{xy} & \mathcal{L}''_{yy} \\
		\end{pmatrix}$$
		Reemplazar para cada punto y si encontrar el determinante, si da positivo es máximo, si da negativo es mínimo. g es la restricción.
		\item Método 2: Inventarse un punto que cumpla la restricción y evaluar todos los puntos en la original, comparar con el inventado para saber cual es mayor y cual es menor.
		
	\end{itemize}
	\item si hay 2 restricciones  $\mathcal{L}=funcion - \lambda_1(restriccion1)-\lambda_2(restriccion2)$ y seguir los mismos pasos.
\end{enumerate}

\subsection{Derivación implícita}
Se usa cuando no dan z despejada

\begin{enumerate}
	\item Igualar a cero y llamar a eso F
	\item Derivar F con respecto a todas las letras $f'_x \qquad f'_y \qquad f'_z$
	\item Encontrar lo que pidan, ejemplo $z'_x=-\dfrac{f'_x}{f'_z}$
\end{enumerate}

\subsection{Integrales dobles (5.3)}

\subsubsection{Cambio de orden de integración, cuando la integral es difícil}

$$\underbrace{\int\overbrace{\int f(x,y)dy}^{funciones}dx}_{Numeros}$$

Si f(x,y)=1, la doble integral es el área

Pasos

\begin{enumerate}
	\item Dibujar región, sacar ecuación de cada límite de integración.
	\item Cambiar orden y límites de integración. $\int\int f(x,y)dydx$
	\item Integrar, puede ser sustitución o partes.
\end{enumerate}

\subsubsection{Teorema del valor medio (5.4)}

$$f(x_0,y_0)=\dfrac{1}{A(D)}\int_Df(x,y)dA$$

\subsection{Integrales triples}
$$\underbrace{\int\overbrace{\int \int f(x,y)dxdy}^{funciones}dz}_{Numeros}$$

Si f(x,y,z)=1, la triple integral es el volumen
\subsection{Cambio de variable}

\begin{enumerate}
	\item Encontrar el Jacobiano
	\begin{itemize}
		\item Si x y y dependen de u y v
		$$
		J=\left|\begin{tabular}{ccc}
		$\dfrac{dx}{du}$ & $\dfrac{dx}{dv}$ \\
		$\dfrac{dy}{du}$ & $\dfrac{dy}{dv}$
		\end{tabular}\right|
		$$
		\item Si u y v dependen de x y y
		$$
		J=\dfrac{1}{\left|\begin{tabular}{ccc}
			$\dfrac{du}{dx}$ & $\dfrac{du}{dy}$ \\
			$\dfrac{dv}{dx}$ & $\dfrac{dv}{dy}$
			\end{tabular}\right|}
		$$
	\end{itemize}
	\item Encontrar ecuaciones para u y v (Despejar si dan x y y en términos de u y v)
	\item Graficar en los plano xy y uv
	\item $\int\int f(u,v)|J|dudv$, el Jacobiano es positivo 
	\item Para polares el Jacobiano es $r$
	
\end{enumerate}





\subsection{Cóncava y convexa}
\subsubsection{Método 1: Hessiana}
\begin{enumerate}
	\item Primeras derivadas
	\item Segundas derivadas
	\item Concluir
	\begin{itemize}
		\item Si $f''_{xx}\geq0$ y $f''_{yy}\geq0$ y $\det(H)\geq0$ es un convexa
		\item Si $f''_{xx}\leq0$ y $f''_{yy}\leq0$ y $\det(H)\geq0$ es un concava
		\item Si $f''_{xx}>0$ y $f''_{yy}>0$ es estrictamente convexa
		\item Si $f''_{xx}<0$ y $f''_{yy}<0$ es estrictamente cóncava
	\end{itemize}
\end{enumerate}

\subsubsection{Método 1: Propiedades}
\begin{enumerate}
	\item Lineal, cóncava o convexa a conveniencia.
	\item Potencias pares $x^2,\ y^4$, convexa
	\item Potencias impares $x^3,\ y^5$, convexa si $x>0$ y cóncava si $x<0$
	\item Raíces pares $\sqrt[2]{x},\ \sqrt[4]{x}$
	\item $e^{algo}$ convexo si algo es convexo
	\item $ln(algo)$ cóncavo si algo es cóncavo
	\item Suma de cóncavas es cóncava.
	\item Suma de convexas es convexa.
	\item Compuesta de convexas es convexa.
	\item Compuesta de cóncavas es cóncava.
	\item -convexa es cóncava
	\item -cóncava es convexa
	\item Cobb-Douglas $x^ay^b$ si $a+b\leq1$ es concava. Si $a+b<1$ es estrictamente cóncava
	\item Si es concava es cuasiconcava
	\item Si es convexa es cuasiconvexa	
\end{enumerate}

Ej: Diga si es cóncava o convexa $3x^{1/4}y^{1/2}z^{1/8}-e^{x^2+y^2+z}+ln(x^2y^3\sqrt{z})$

