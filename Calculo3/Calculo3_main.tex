\section{CÁLCULO 3}

\subsection{Funciones}
\begin{itemize}
	\item Escalares: Al reemplazar dan un número
	\item Vectoriales: Al reemplazar dan un vector
\end{itemize}

\subsection{Representación geométrica}

\subsubsection{Curvas de nivel}
\begin{enumerate}
	\item Cambiar z por k
	\item Despejar y (pero si hay $x^2+y^2$, despejar eso porque es un círculo)
	\item Darle valores a k, al menos 3
	\item Graficar para cada valor de k
	\item Graficar en 3D, para saber corte con z, reemplazar x y y por 0
\end{enumerate}

\subsubsection{Gráficas importantes}
\begin{itemize}
	\item Esfera: $x^2+y^2+z^2=\rho^2$, radio=$\rho$
	\item Cilindro:  $x^2+y^2=r^2$, la letra que no aparece es hacia donde abre
	\item Cono: $ z=\sqrt{x^2+y^2}$, la letra que no está en la raiz es hacia donde abre.
	\item Paraboloide: $ z=x^2+y^2$, el que no tiene el cuadrado es hacia donde abre.
	\item Elipsoide: $\dfrac{x^2}{a^2}+\dfrac{y^2}{b^2}+\dfrac{z^2}{c^2}=1$, a, b y c son los puntos donde corta en x, y y z respectivamente.
\end{itemize}
\subsection{Derivadas Parciales}
\begin{itemize}
	\item Primeras derivadas: $f'_x=\dfrac{\delta f}{\delta x}$ $f'_y=\dfrac{\delta f}{\delta y}$
	\item Segundas derivadas: $f''_{xx}=\dfrac{\delta^2 f}{\delta x^2}$, $f''_{yy}=\dfrac{\delta^2 f}{\delta y^2}$, $f''_{xy}=\dfrac{\delta^2 f}{\delta x\delta y}$, $f''_{yx}=\dfrac{\delta^2 f}{\delta y\delta x}$
	\item Las cruzadas son iguales por el teorema de Young
\end{itemize}
\subsection{Ecuación plano tangente}

\begin{itemize}
	\item Si la función es de dos variables
	$$f'_x(x_0,y_0)(x-x_0)+f'_y(x_0,y_0)(y-y_0)=z-z_0$$
	\item Si la función es de 3 variables
	$$f'_x(x_0,y_0)(x-x_0)+f'_y(x_0,y_0)(y-y_0)+f'_z(x_0,y_0)(z-z_0)=0$$	
\end{itemize}

\textbf{Pasos}

\begin{enumerate}
	\item Encontrar las primeras derivadas.
	\item Reemplazar el punto en las derivadas.
	\item Reemplazar en ecuación del plano.
\end{enumerate}

\subsection{Formas cuadráticas}
\begin{enumerate}
	\item Encontrar las primeras derivadas
	\item Encontrar las segundas derivadas
	\item Matriz Hessiana o matriz simétrica
	
	H=$\begin{pmatrix}
	f''_{xx} & f''_{xy} \\
	f''_{xy} & f''_{yy} \\
	\end{pmatrix}$
	
	H=$\begin{pmatrix}
	f''_{xx} & f''_{xy} & f''_{xz} \\
	f''_{xy} & f''_{yy} & f''_{yz}\\
	f''_{xz} & f''_{yz} & f''_{zz}\\
	\end{pmatrix}$
	\item Conclusión
	\begin{itemize}
		\item Si los dominantes son todos positivos, es definida positiva
		\item Si los dominantes son de signos alternados empezando con -, es difinida negativa.
		\item Si los dominantes son $\geq 0$ (Puede ser cero), es semidefinda positiva.
		\item Si los dominantes son $\leq 0$ (Puede ser cero), es semidefinida negativa.
	\end{itemize}
\end{enumerate}

\subsection{Regla de la cadena}
Si F(u,v) y u(x,y) v(x,y)
\begin{itemize}
	\item $f'_x=F'_uu'_x+F'_vv'_x$
	\item $f'_y=F'_uu'_y+F'_vv'_y$
\end{itemize}

\subsection{Derivación implícita}
Se usa cuando no dan z despejada

\begin{enumerate}
	\item Igualar a cero y llamar a eso F
	\item Derivar F con respecto a todas las letras $f'_x \qquad f'_y \qquad f'_z$
	\item Encontrar lo que pidan, ejemplo $z'_x=-\dfrac{f'_x}{f'_z}$
\end{enumerate}

\subsection{Derivada Direccional}

$$\overrightarrow{D}_f=f'_x(x_0,y_0)h+f'_y(x_0,y_0)k$$
$$\overrightarrow{D}_f=\nabla f \cdot V$$

Derivada direccional de f en al punto $(x_0,y_0)$ en la dirección del vector unitario (h,k).

\subsection{Elasticidades}

Elasticidad de z con respecto a x $el_xZ=\dfrac{x}{z}z'_x$

\textbf{Tasa marginal de sustitución}

$R_{yx}=\dfrac{f'x}{f'y}$ y $R_{xy}=\dfrac{f'y}{f'x}$

\textbf{Elasticidad de sustitución $\sigma_{yx}$}

$\sigma_{yx}=el_{R_{yx}}(\dfrac{y}{x})=\dfrac{R_{yx}}{\dfrac{y}{x}}(\dfrac{y}{x})'_R$

$\sigma_{xy}=el_{R_{xy}}(\dfrac{x}{y})=\dfrac{R_{xy}}{\dfrac{x}{y}}(\dfrac{x}{y})'_R$

\textbf{Pasos para $\sigma_{yx}$ o $\sigma_{xy}$}

\begin{enumerate}
	\item Derivar
	\item Encontrar $R_{yx}$ o $R_{xy}$
	\item Despejar $\dfrac{y}{x}$ o $\dfrac{x}{y}$
	\item Derivar $\dfrac{y}{x}$ o $\dfrac{x}{y}$ respecto a R
	\item Reemplazar en la formula cambiando todo menos R. Debe dar un número
\end{enumerate}

\subsection{Funciones Homogeneas}
Una función es homogenea si $f(tx,ty)=t^nf(x,y)$, homogenea de grado n
\begin{enumerate}
	\item Cambiar x por tx y y por ty
	\item Romper paréntesis
	\item Sacar factor común t
	\item Si se llega a $t^nf(x,y)$ es homogenea.
\end{enumerate}

\textbf{Teorema de Euler}

Si $xf'_x+yf'_y=kf(x,y)$ homogénea de grado k.

\subsection{Aproximaciones lineales}

$f(x,y)=f(a,b)+f'_x(a,b)(x-a)+f'_y(a,b)(y-b)$, son los mismos pasos del plano

\subsection{Sistemas de ecuaciones}

\begin{enumerate}
	\item Derivar respecto a todo y cada vez que se deriva una letra se pone dletra.
	\item Pasar todo lo que tiene du y dv al lado izquierdo y pasar todo lo que tiene dx y dy al lado derecho
	\item Organizar en una matriz
	\item Calcular det(A)
	\item Encontrar lo que piden usando cramer
\end{enumerate}

\subsection{Integrales dobles}

\subsubsection{Cambio de orden de integración, cuando la integral es difícil}

$$\underbrace{\int\overbrace{\int f(x,y)dy}^{funciones}dx}_{Numeros}$$

Pasos

\begin{enumerate}
	\item Dibujar región, sacar ecuación de cada límite de integración.
	\item Cambiar orden y límites de integración. $\int\int f(x,y)dydx$
	\item Integrar, puede ser sustitución o partes.
\end{enumerate}

\subsection{Cambio de variable}

\begin{enumerate}
	\item Encontrar el Jacobiano
	\begin{itemize}
		\item Si x y y dependen de u y v
		$$
		J=\left|\begin{tabular}{ccc}
		$\dfrac{dx}{du}$ & $\dfrac{dx}{dv}$ \\
		$\dfrac{dy}{du}$ & $\dfrac{dy}{dv}$
		\end{tabular}\right|
		$$
		\item Si u y v dependen de x y y
		$$
		J=\dfrac{1}{\left|\begin{tabular}{ccc}
			$\dfrac{du}{dx}$ & $\dfrac{du}{dy}$ \\
			$\dfrac{dv}{dx}$ & $\dfrac{dv}{dy}$
			\end{tabular}\right|}
		$$
	\end{itemize}
	\item Encontrar ecuaciones para u y v (Despejar si dan x y y en términos de u y v)
	\item Graficar en los plano xy y uv
	\item $\int\int f(u,v)|J|dudv$, el Jacobiano es positivo 
	
\end{enumerate}



\subsection{Máximos y mínimos sin región}
\begin{enumerate}
	\item Encontrar derivadas parciales con respecto a x y y, $f'_{x}$ y $f'_{y}$.
	\item Igualar a cero y despejar las variables, esos son los puntos críticos o estacionarios.
	\item Encontrar las segundas derivadas $f''_{xx}$, $f''_{yy}$ y $f''_{xy}$
	\item Encontrar $\triangle$
	
	H=$\begin{pmatrix}
	f''_{xx} & f''_{xy} \\
	f''_{xy} & f''_{yy} \\
	\end{pmatrix}$
	
	$\triangle=\det H$ o $\triangle = f''_{xx}f''_{yy}-(f''_{xy})^2$
	\item Conclusión 
	\begin{itemize}
		\item Si $f''_{xx}>0$ y $\triangle>0$ es un mínimo
		\item Si $f''_{xx}<0$ y $\triangle>0$ es un máximo
		\item Si $\triangle<0$ es un punto de silla
	\end{itemize}
\end{enumerate}

Ej: $f(x,y)=x^3+y^3-9xy+27$ en (0,0) (0,1) (1,1)

\subsection{Máximos y mínimos con región}

\begin{enumerate}
	\item Encontrar derivadas parciales con respecto a x y y, $f'_{x}$ y $f'_{y}$.
	\item Igualar a cero y despejar las variables, esos son los puntos críticos o estacionarios.
	\item Encontrar las segundas derivadas $f''_{xx}$, $f''_{yy}$ y $f''_{xy}$
	\item Dibujar región
	\item Nombrar los bordes de la región.
	\item Calcular para cada borde una función, reemplazando el borde en la original.
	\item Derivar e igualar a cero cada función y encontrar puntos
	\item Escoger puntos que están en la región y las esquinas (En un círculo las esquinas son los cortes con el eje x).
	\item Reemplazar todos los puntos en la original.
	\item Verificar cual es el máx y cual es el mín absoluto. 
\end{enumerate}

\subsection{Cóncava y convexa}
\subsubsection{Método 1: Hessiana}
\begin{enumerate}
	\item Primeras derivadas
	\item Segundas derivadas
	\item Concluir
	\begin{itemize}
		\item Si $f''_{xx}\geq0$ y $f''_{yy}\geq0$ y $\det(H)\geq0$ es un convexa
		\item Si $f''_{xx}\leq0$ y $f''_{yy}\leq0$ y $\det(H)\geq0$ es un concava
		\item Si $f''_{xx}>0$ y $f''_{yy}>0$ es estrictamente convexa
		\item Si $f''_{xx}<0$ y $f''_{yy}<0$ es estrictamente cóncava
	\end{itemize}
\end{enumerate}

\subsubsection{Método 1: Propiedades}
\begin{enumerate}
	\item Lineal, cóncava o convexa a conveniencia.
	\item Potencias pares $x^2,\ y^4$, convexa
	\item Potencias impares $x^3,\ y^5$, convexa si $x>0$ y cóncava si $x<0$
	\item Raíces pares $\sqrt[2]{x},\ \sqrt[4]{x}$
	\item $e^{algo}$ convexo si algo es convexo
	\item $ln(algo)$ cóncavo si algo es cóncavo
	\item Suma de cóncavas es cóncava.
	\item Suma de convexas es convexa.
	\item Compuesta de convexas es convexa.
	\item Compuesta de cóncavas es cóncava.
	\item -convexa es cóncava
	\item -cóncava es convexa
	\item Cobb-Douglas $x^ay^b$ si $a+b\leq1$ es concava. Si $a+b<1$ es estrictamente cóncava
	\item Si es concava es cuasiconcava
	\item Si es convexa es cuasiconvexa	
\end{enumerate}

Ej: Diga si es cóncava o convexa $3x^{1/4}y^{1/2}z^{1/8}-e^{x^2+y^2+z}+ln(x^2y^3\sqrt{z})$

\subsection{Lagrange}

Si preguntan condiciones necesarias y suficientes. Las condiciones necesarias son las primeras derivadas $\mathcal{L}_x$, $\mathcal{L}_y$ y $\mathcal{L}_\lambda$ y las condiciones suficientes es que es que el punto que se encuentre sea máximo o mínimo de acuerdo a lo que se está pidiendo.

\begin{enumerate}
	\item Plantear Lagrangiano, $\mathcal{L}=funcion - \lambda(restriccion)$. La restricción debe estar igualada a cero.
	\item Encontrar $\mathcal{L}_x$, $\mathcal{L}_y$ y $\mathcal{L}_\lambda$
	\item Igualar a cero todas las derivadas y despejar $\lambda$ de cada una.
	\item Igualar los $\lambda$ y encontrar alguna variable.
	\item Reemplazar esa variable en la restricción y encontrar todas las variables.
	\item Verificar que cumple.
	\begin{itemize}
		 \item Método 1: Si la función es concava es un máximo. Si la función es convexa es un mínimo.
		 \item Método 2: Hessiano ampliado (orlado)
		 
		 $$H=\begin{pmatrix}
		 0 & g'_x & g'_y \\
		 g'_x & \mathcal{L}''_{xx} & \mathcal{L}''_{xy} \\
		 g'_y & \mathcal{L}''_{xy} & \mathcal{L}''_{yy} \\
		 \end{pmatrix}$$
		 Reemplazar para cada punto y si da positivo es máximo, si da negativo es mínimo. g es la restricción.
		 \item Método 3: si $\mathcal{L}$ es concava es un máximo, si es convexa es un mínimo.
		 \item Método 4: Inventarse un punto que cumpla la restricción y evaluar todos los puntos en la original, comparar con el inventado para saber cual es mayor y cual es menor.
		 
	\end{itemize}
	\item si hay 2 restricciones  $\mathcal{L}=funcion - \lambda_1(restriccion1)-\lambda_2(restriccion2)$ y seguir los mismos pasos.
\end{enumerate}