\section{PRECÁLCULO}

\subsection{Capítulo 1}
\subsubsection{1.5}
\subsubsection{1.6}
\subsubsection{1.7}
\subsubsection{1.8}
\subsubsection{1.10}
\subsection{Capítulo 2}
\subsubsection{2.1 - Función}
Una función es la relación entre 2 o mas variables (velocidad-tiempo)\\
f (x) se lee “f de x”\\
Dominio: Es el conjunto de todos los valores que puede tomar x o la variable independiente\\
\begin{itemize}
	\item Polinomios
	\item Raíces
	\item Polinomio sobre polinomio
	\item Polinomio sobre Raíz
	\item Raíz sobre polinomio
\end{itemize}
Rango: Es el conjunto de todos los valores que puede tomar y o la variable dependiente 
\subsubsection{2.2 - Gráficas}
Tabular sabiendo que cada punto es (x, f (x))\\
Funciones por tramos\\
Prueba de la recta vertical\\
Funciones básicas
\begin{itemize}
	\item Funciones lineales
	\item Funciones de potencia
	\item Funciones de raíz (par e impar)
	\item Funciones recíprocas (par e impar)
	\item Función valor absoluto
	\item Función parte entera
\end{itemize}
\subsubsection{2.5 - Transformaciones}
Transformaciones\\
Funciones pares e impares
\subsubsection{2.6 - Combinación de funciones}
Suma, resta, multiplicación, división (Dominio es la intersección excepto en división que se deben quitar los puntos donde el denominador es cero)\\
Composición
\subsubsection{2.7 - Funciones uno a uno e inversas}
Funciones uno a uno f (x1) = f (x2), prueba de la recta horizontal.\\
Inversa, dominios y rangos\\
La gráfica de la inversa es intercambiar los puntos de la original
\subsection{Capítulo 3}
\subsubsection{3.1 - Funciones y modelos cuadráticos}
Forma general y canónica, vértice (mínimo y máximo), completación de cuadrados
\subsubsection{3.3 - División de polinomios}
\subsubsection{6.1 - Medida de un ángulo}
\subsubsection{6.2 - Trigonometría de triángulos rectángulos}
\subsubsection{6.3 - Funciones trigonométricas de ángulos}
\subsubsection{5.3 - Gráficas trigonométricas}
\subsubsection{5.4 - Más gráficas trigonométricas}
\subsubsection{5.5 - Funciones trigonométricas inversas y sus gráficas}
\subsubsection{6.4 - Funciones trigonométricas inversas y triángulos rectángulos:}
\subsubsection{7.2 Fórmulas de adición y sustracción}
\subsubsection{7.3 Fórmulas de ángulo doble, semiángulo y producto a suma}
\subsubsection{7.4 Ecuaciones trigonométricas}



\begin{itemize}
	\item 
\end{itemize}