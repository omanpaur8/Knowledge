\section{DIFERENCIAL}

\subsection{Dominio}

\subsection{Función Inversa}

\subsection{Límites laterales}

\subsection{Límites}

\subsection{Continuidad}

\subsection{Definición de derivada}

\begin{align*}
	\lim\limits_{h \to 0}& \dfrac{f(x+h)-f(x)}{h}\\
	\lim\limits_{x \to a}& \dfrac{f(x)-f(a)}{x-a}
\end{align*}

\subsection{Propiedades de derivadas}
\begin{itemize}
	\item $(f\pm g)'=f'\pm g'$
	\item $(f\cdot g)'=f'g+fg'$
	\item $(f/g)'=\dfrac{f'g-fg'}{g^2}$
	\item $f(g(x))'=f'(g(x))g'(x)$
\end{itemize}

\subsection{Lista de derivadas}
\begin{table}[H]
	\centering

\begin{tabular}{ll}
	\hline \hline
	Función & Integral\\
	\hline
	$\#$ & $0$\\
	$\#x$ & $\#$\\
	$x$ & $1$\\
	$x^n$ & $(n+1)x^{n-1}$\\
	$\ln(x)$ & $\dfrac{1}{x}$\\
	$\ln(x)$ & $\dfrac{1}{x}$\\
	$e^x$ & $e^x$\\
	$\#^x$ & $\#^x\ln(\#)$\\
	$sin(x)$ & $cos(x)$\\
	$cos(x)$ & $-sen(x)$\\
	$tan(x)$ & $sec^2(x)$\\
	$tan(x)$ & $sec(x)tan(x)$\\
	$cot(x)$ & $-csc^2(x)$\\
	$csc(x)$ & $-csc(x)cot(x)$\\
	$arctan(x)$ & $\dfrac{1}{1+x^2}$\\
	$arcsen(x)$ & $\dfrac{1}{\sqrt{1-x^2}}$\\
	$arcsec(x)$ & $\dfrac{1}{x\sqrt{x^2-1}}$\\
	\hline
\end{tabular}

\end{table}


\subsection{Regla de la cadena}

\subsection{Derivación implícita}

\subsection{Derivación Logarítmica}

\subsection{Máximos y mínimos}

\subsection{Ejercicios claves segundo Parcial}


\begin{itemize}
	\item Si piden ecuación de recta tangente y dan un punto (x,y).
	\begin{itemize}
		\item Si al reemplazar el x en la función da el y del punto
		\begin{enumerate}
			\item Derivar
			\item Reemplazar el punto. Lo que da es m
			\item Encontrar la ecuación de la recta.
		\end{enumerate}
	\item Si al reemplazar el x en la función no da el y del punto, es decir, el punto no está en la función.
	\begin{enumerate}
		\item Elegir un punto imaginario en la función, (a,f(a)). Para encontrar f(a), reemplazar a en la original.
		\item Encontrar $m=\dfrac{y_2-y_1}{x_2-x_1}$ con el punto que dan y el punto imaginario
		\item Encontrar $f'(a)$, es decir derivar y reemplazar a.
		\item Hacer $f'(a)=m$ y despejar a.
		\item Con los valores de a, reemplazar en la derivada para encontrar pendientes
		\item Encontrar ecuaciones de rectas.
	\end{enumerate}
	\end{itemize}
	
	\subsection{Teorema del valor medio}
	\begin{itemize}
		\item Si f es una función continua en [a,b] y
		\item f' es derivable en (a,b), es decir que la derivada es continua en (a,b)
		\item existe un valor c tal que $f'(c)=\dfrac{f(b)-f(a)}{b-a}$
	\end{itemize}
	\subsection{Teorema de Rolle}
	\begin{itemize}
		\item Si f es una función continua en [a,b] y
		\item f' es derivable en (a,b), es decir que la derivada es continua en (a,b)
		\item $f(a)=f(b)$
		\item existe un valor c tal que $f'(c)=0$
	\end{itemize}
	\subsection{Raíces en un intervalo}
	\begin{itemize}
		\item Encontrar 2 números para los cuales al reemplazar en la función uno da negativo y otro positivo. 
		\item Verificar si la función es continua [a,b].
		\item Por el teorema del valor intermedio se garantiza la existencia de al menos una raiz o un cruce por el eje x en el intervalo (a,b).
		\item Verificar si la función es derivable en (a,b).
		\item Encontrar la derivada. Si no hay puntos críticos, entonces la derivada es siempre positiva o siempre negativa por lo que solo hay una raíz en (a,b).
	\end{itemize}
\end{itemize}

\subsection{Trazo de curvas}
\begin{enumerate}
	\item Dominio
	\item Cortes con ejes
	\item Simetría
	\item Asíntotas
	\begin{itemize}
		\item Verticales
		\item Horizontales
		\item Oblicuas: Son de la forma y=mx+b
		$m=\lim\limits_{x \to \infty}\frac{f(x)}{x}$ y $b=\lim\limits_{x \to \infty}f(x)-mx$
	\end{itemize}
	\item Intervalos de crecimiento y decrecimiento
	\item Máximos y mínimos
	\item Concavidad
	\item Bosquejo
	\item Rango
\end{enumerate}

\subsection{Tasas de Cambio}

\subsection{Optimización}

\begin{enumerate}
	\item Anotar datos y hacer dibujo, identificar restricción (valor o ecuación que dan).
	\item Encontrar función objetivo, es lo que se quiere maximizar o minimizar.
	\item Encontrar ecuación de restricción.
	\item Despejar una letra de la restricción y reemplazar en la función objetivo.
	\item Derivar, igualar a cero y despejar.
	\item Comprobar los puntos encontrados con cementerio o reemplazando en le segunda derivada.
	\item Encontrar lo que piden.
\end{enumerate}

\subsection{L'Hopital}

\begin{tabular}{@{}|p{2cm}| @{}|p{2cm}| @{}|p{2.5cm}| @{}|p{4cm}|}
	\hline \hline
	$\dfrac{0}{0},\ \dfrac{\infty}{\infty}$  & $0\cdot \infty$  & $\infty-\infty$ & $1^\infty,\ \infty^0,\ 0^0$ \\
	\hline
	Derivar arriba y abajo & Bajar una función, no bajar LN  & Si son fracciones sumar, si son raices racionalizar, si son logaritmos $ln(a)-ln(b)=ln(\dfrac{a}{b})$ & \begin{itemize}
		\item Poner $y=lim$
		\item sacar ln a ambos lados $ln(y)=lim ln()$
		\item Bajar exponente
		\item Aplicar algún l'hopital
		\item Respuesta final, poner $y=e^{rta}$
		\end{itemize}
	\\
	\hline
\end{tabular}

\subsection{Área entre curvas}

\begin{enumerate}
	\item Igualar ecuaciones y resolver para encontrar puntos de corte (x,y)
	\item Graficar (Transformaciones o tabular)
	\item Plantear la integral
	\item $A=\int(\text{función mayor - función menor})$
\end{enumerate}

\subsection{Sólidos de Revolución}

\begin{enumerate}
	\item Igualar ecuaciones y resolver para encontrar puntos de corte (x,y)
	\item Graficar solo la región
	\item Pintar eje de rotación y dibujar espejo
	\item Encontrar volumen
	\item Arandelas $V=\pi\int_{a}^{b}(R)^2-(r)^2$
	\begin{itemize}
		\item Pintar arandelas, radio mayor y radio menor.
		\item Si gira en eje vertical
		\begin{itemize}
			\item El diferencial es dy
			\item a y b se ven en y
			\item los radios se calculan como eje -función (función en términos de y, se despeja x)
			\begin{itemize}
				\item eje -función, si eje está por derecha
				\item función - eje, si eje está por izquierda
			\end{itemize}
		\end{itemize}
		\item Si gira en eje horizontal
		\begin{itemize}
			\item El diferencial es dx
			\item a y b se ven en x
			\item los radios se calculan como (función en términos de x, se despeja y)
			\begin{itemize}
				\item eje -función, si eje está por arriba
				\item función - eje, si eje está por abajo
			\end{itemize}
		\end{itemize}
	\end{itemize}
	\item Casquetes cilíndricos $V=2\pi\int_{a}^{b}(\text{H})(radio)$
	\begin{itemize}
		\item Pintar cilindro, radio y altura.
		\item Si gira en eje vertical
		\begin{itemize}
			\item Se pone dx
			\item a y b se ven en x
			\item H=Techo - piso
			\item el radio se calcula como:
			\begin{itemize}
				\item eje -x, si eje está por derecha
				\item x - eje, si gráfica está por izquierda
			\end{itemize}
		\end{itemize}
		\item Si gira en eje horizontal
		\begin{itemize}
			\item Se pone dyg
			\item a y b se ven en y
			\item H=Derecha - izquierda
			\item los radio se calculan como:
			\begin{itemize}
				\item eje -y, si eje está por arriba
				\item y - eje, si eje está por abajo
			\end{itemize}
		\end{itemize}
	\end{itemize}
\end{enumerate}

\subsection{Interés compuesto}
\subsubsection{Tasas efectivas}
$$A=P(1+i)^n$$
\begin{itemize}
	\item A: Capital, valor futuro, ahorro
	\item P: Inversión inicial
	\item n: Número de periodos
	\item i: Tasa de interés
\end{itemize}

\subsubsection{Tasas nominales, capitalizables, convertibles}
$$A=P(1+i/n)^{nt}$$
\begin{itemize}
	\item A: Capital, valor futuro, ahorro
	\item P: Inversión inicial
	\item n: Número de periodos
	\item i: Tasa de interés
\end{itemize}

\subsubsection{Interés continuo}
$$A=Pe^{it}$$
\begin{itemize}
	\item A: Capital, valor futuro, ahorro
	\item P: Inversión inicial
	\item i: Tasa de interés
\end{itemize}

\subsection{Crecimiento logístico}
$$f(t)=\dfrac{L}{Ab^{-t}}$$
\begin{itemize}
	\item L: Límite
\end{itemize}

\subsection{Curva de aprendizaje}
$$f(t)=L-Ab^{-t}$$
\begin{itemize}
	\item L: Límite
\end{itemize}

\subsection{Modelos log y exp}
$$f(t)=ab^{t}$$
\begin{itemize}
	\item a: Valor para t=0
\end{itemize}

$$f(t)=\log_ax$$
\begin{itemize}
	\item a: Valor de x cuando la gráfica vale 1
\end{itemize}

\subsection{Elasticidad}
$$e=\dfrac{-p}{q}q'$$
\begin{itemize}
	\item $|e|>1$ Elástica
	\item $|e|<1$ Inelástica
	\item $|e|>1$ Perfectamente elástica
	\item Para un incremento del $1\%$ del precio la demanda disminuye en ...
\end{itemize}

\subsection{Bienes sustitutos y complementarios}

\begin{itemize}
	\item Si las derivadas parciales cruzadas con positivas los bienes son sustitutos
	\item Si las derivadas parciales cruzadas con negativas los bienes son complementarios
\end{itemize}

