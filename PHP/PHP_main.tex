\section{PHP}

Acrónimo recursivo de PHP: Pre Hypertext Preprocessor.  Es un lenguaje de código abierto muy popular especialmente adecuado para el desarrollo web y que puede ser incrustado en HTML. Se ejecuta del lado del servidor generando HTML y enviándolo al cliente.

\subsection{Printing}

\begin{itemize}
	\item echo: it command allows to print several results. \textit{echo: $""$, $""$;}
	\item print: it command  allows to print one result at a time. \textit{print: "";}
\end{itemize}


\subsection{Variables}

The name of the variable must begin with \$ simbol.
\begin{itemize}
	\item \$numeric = 5;
	\item \$text = "string";
	\item \$booleana = true;
	\item \$arrays = array($"element1","element2"$); to index the array it must be used squared pharentesis and the index begins in zero.
	\item \$objects= (object)[$"element1"=>"value1","element2"=>"value2"$]; to access de object objects=>elemen1
\end{itemize}

\subsection{Methods}

\begin{itemize}
	\item var\_dump(variable) the method allows to know the type of variable it's being used and the content or variable length.
\end{itemize}

\subsection{Functions}
The sintaxes of a function is
\begin{lstlisting}
	function (parameters){
		return var;
	}
\end{lstlisting}

\subsection{Conditionals}
The sintaxes of a conditional is
\begin{lstlisting}
	if(condition){
		code
	}else if(condition){
		code
	}else{
		code
	}
\end{lstlisting}

The sintaxes of a switch case is

\begin{lstlisting}
	switch(varCase){
		case :
		break;
		
		case :
		break;
		
		default:
		break;
	}
\end{lstlisting}

\subsection{cycles}

The sintaxes of a while cycle is

\begin{lstlisting}
	while(condition){
		code
	}
\end{lstlisting}

The sintaxes of a do while cycle is

\begin{lstlisting}
	do{
		code
	}while(condition);
\end{lstlisting}

The sintaxes of a for cycle is

\begin{lstlisting}
	for($i=0; $i<5; $i++){
		code
	}
\end{lstlisting}

